\chapter{Introduction} \label{sec:introduction}
Computers have been successfully applied to many complex problems, such as those in finance, healthcare, and the Internet. However, computers still struggle with \hl{several} problems. Consider whether a given program on any input will always terminate. One can easily show that certain programs will always halt. For example, we can easily show a program that takes an integer input $x$, computes $y=x+1$, prints $y$, then halts, will always terminate. Nevertheless, there exist simple programs that are much harder to determine if they always terminate. Algorithm~\ref{alg:ColR} is such an example. Will this program always return 1 for any positive integer $N$, causing it to halt? This problem is a reformulation of the Collatz Conjecture, \hl{also known as the $3N+1$ problem.} \par
\begin{algorithm} 
\caption{The Collatz Conjecture Sequence, $\Col{N}$}
\label{alg:ColR} 
\begin{algorithmic}[1]
    \If{$N \leq 1$} \Return $N$ 
    \EndIf
    \If {$N \equiv 0 \Mod{2}$} \Return $\Col{N/2}$
    \EndIf
    \State \Return $\Col{3N + 1}$ 
\end{algorithmic}
\end{algorithm}
We know from Turing that no program exists to determine if an arbitrary program with arbitrary input can halt~\cite{Turing1936}, but that does not exclude the possibility of a program that determines if specifically Algorithm~\ref{alg:ColR} halts. However, no program has been found to show that all positive integer inputs for Algorithm~\ref{alg:ColR} will halt, even though this problem can be explained to an elementary grade student. The problem has been extensively analyzed, according to surveys by Lagrias~\cite{2003mathLagrais}~\cite{2006mathLagrias}, yet no proof has been found. The search is further motivated by extensive empirical evidence suggesting that the Collatz Conjecture is true. According to a website maintained by Roosendahl~\cite{EricRoose}, all numbers up to $87 \cdot 2^{60}$, or about $10^{20}$, have been tried as $N$ in Algorithm~\ref{alg:ColR}, and have converged to 1. \par
Since the Collatz Conjecture has been challenging to prove, we propose another supposedly simpler variant of it. Can we prove that the code in Algorithm \ref{alg:ColSP}, where $A = \{1\}$, and $b = 8$, always terminates for any positive integer $N$? Even though this program seems to be easier, we do not have a program showing this variant always terminates either! \par
\begin{algorithm} 
\caption{A Collatz Conjecture Variant $\ColMod{N}{A}{b}$}
\label{alg:ColSP} 
\begin{algorithmic}[1]
    \If{$(N \leq 1) \vee (N \equiv a_1  \Mod{b}) \vee \ldots \vee (N \equiv a_s \Mod{b})$ } \Return $N$
    \EndIf
    \If {$N \equiv 0 \Mod{2}$} \Return $\ColMod{N/2}{A}{b}$
    \EndIf
    \State \Return $\ColMod{3N+1}{A}{b}$ 
\end{algorithmic}
\end{algorithm}
One of the goals of this thesis is to try and determine how hard certain variants of the Collatz Conjecture are to solve. A contribution of this thesis is that it uses empirical data to try and find trends of hardness for difficult variants, and compare these trends to the hardness of solving the whole Collatz Conjecture. This thesis also follows an approach that Heule and Aaronson have devised attempting to craft a program to determine if Algorithm~\ref{alg:ColR} always halts~\cite{HeuleAaronson}. At a high level, it involves taking a completely reworked formulation of Algorithm~\ref{alg:ColR} and using known techniques that, if certain conditions are met, the reworked formulation can be shown to terminate for any positive integer input. The formulation requires SAT solvers, string rewrite systems, and a technique called matrix interpretation, all topics which will be covered briefly in this paper as background. This thesis also investigates a rewrite system that Aaronson crafted, and we believe that, if Aaronson's system is found to terminate for all input, the Collatz Conjecture holds. We analyze properties of this rewrite system, and the previously variants are investigated with this rewrite system as well.\par
The rest of this thesis is outlined as follows: Chapter~\ref{sec:defns} introduces definitions that will be used throughout the paper. Chapter~\ref{sec:alttercdns} defines several Collatz Conjecture Variants, both solved and unsolved, including $\ColMod{N}{\{1\}}{8})$. Chapter~\ref{sec:subhrdnspred} analyzes the difficulty of these variants using algebra. Chapter ~\ref{sec:SRSandSAT} \hl{discusses the results so far of Heule using Aaronson's rewrite system and parallel SAT solving to prove the Collatz Conjecture and hard Collatz Variants, as well as necessary background to understand the approach.} Finally, Chapter~\ref{sec:hardnessrewriterules} investigates hardness of solving the same variants covered in Chapter~\ref{sec:subhrdnspred}, but derived from Aaronson's rewrite system instead.
%%  LocalWords:  healthcare Lagrias
