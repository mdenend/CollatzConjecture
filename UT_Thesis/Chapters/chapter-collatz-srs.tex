\chapter{The Collatz Conjecture as a rewriting system} \label{sec:CollatzSRS}
Using the knowledge discussed in section~\ref{sec:SRSandSAT}, Aaronson built an SRS representing the Collatz Conjecture~\cite{HeuleAaronson}. We'll call this the Collatz SRS throughout the remainder of this Thesis. \par
Let the alphabet of the Collatz SRS consist of the symbols $a, b, c, d, e, f, g$. The symbols can be written as these linear functions:
\[
a(x) = 2x, b(x) = 2x+1, c(x) = 1, d(x) = x, e(x) = 3x, f(x) = 3x+1, g(x) = 3x+2
\]
%the binary symbols and ternary symbols part is a bit awkward, but decided to come back to it later.
The symbols $a$ and $b$ are binary symbols. They represent  a binary system: $a$ is 0 and $b$ is 1.  The symbols $e$, $f$, and $g$ are ternary symbols. They represent a ternary system: $e$ is 0, $f$ is 1, and $g$ is 2. $c$ and $d$ are placeholder symbols to represent the leading 1 and the end of the string, respectively. They help this rewrite system know where the beginning and end of the string are. \par
Note that in order to correctly compute the values that the strings represent using the above linear functions, one needs to read the functions from left to right. That is to say, the string $cabad$, which is equal to 10, is not equal to $ c \circ a \circ b \circ a \circ d$, where $\circ$ is the composition of functions. Instead, $cabad = d \circ a \circ b \circ a \circ c$. Aaronson chose to write the strings like this since they follow the way we would write numbers. \par
Using the provided alphabet, Aaronson created the following series of SRRs:
\begin{align*}
    D_1 : ad &\rightarrow d\ & \ A_1 : ae &\rightarrow ea\ & \ B_1 : be &\rightarrow fb\ & \ C_1 : ce &\rightarrow cb \\
    D_2 : bd &\rightarrow gd\ & \ A_2 : af &\rightarrow eb\ & \ B_2 : bf &\rightarrow ga\ & \ C_2 : cf &\rightarrow caa \\
    &\ &\ A_3 : ag &\rightarrow fa\ &\ B_3 : bg &\rightarrow gb\ &\ C_3 : cg &\rightarrow cab
\end{align*}
The SRRs provided here allow for Aaronson's SRS to be equivalent to the $3x+1$ mapping, but a formal proof showing this is the case is difficult, because we have yet to find a proof showing that the rewrite rules can be applied in arbitrary order. However, we can explain how the rules work, and show that any valid input string correctly follows the $3x+1$ mapping for the number the string represents, and after this, we will show the ordering we follow, and why this ordering is correct. \par
Each of the rules denotes how to handle the symbols $a-g$, or the combination of binary and placeholder strings. The $D$ rules represent the two steps taken by the Collatz Conjecture in a binary system. $D_1$ is how we handle an even number. It is equivalent to $x >> 1$, or divison of $x$ by 2. $D_2$ is actually a combination of several steps. If we were to represent $3x+1$, we could just write $bd \rightarrow bfd$, meaning take all previous symbols and multiply the result by 3 and add 1. The problem with this rule is that it increases the size of the resulting string, making the system more difficult to prove. However, $bfd \rightarrow gad$ is a valid rule,  as $d \circ f \circ b = 3(2x+1)+1 = 6x+4$ and $d \circ a \circ g = 2(3x+2) = 6x+4$, and from here, we can apply the rule $ad \rightarrow d$ to allow us to do $gad \rightarrow gd$. Since $3x+1$ always results in an even number, we can just make rule $D_2$ compute $(3x+1)/2$ without growing the string size, ultimately making rule $D_2$ into $bd \rightarrow gd$. $D_2$ is the rule that makes termination of our system hard to prove. Without it, we would not need the $A$, $B$, or $C$ rules.\par
The $A$, $B$ and $C$ rules all deal with the handling of the ternary symbols and the eventual conversion of these ternary symbols into the binary symbols. The $A$ and $B$ rules deal with the case when a ternary symbol is to the right of the binary symbol, and how to switch the ternary symbol and the binary symbol without changing the number the string represents. We will show that all 6 of these rules preserve the same number by showing that the string represents the same value after each rule has been applied:
\begin{itemize}
    \item $\boldsymbol{ae \rightarrow ea}$: $ae = e \circ a = e(a(x)) = 2(3x) = 6x$, and $ea = a
    \circ e = a(e(x)) = 3(2x) = 6x$.
    \item $\boldsymbol{af \rightarrow eb}$: $af = f \circ a = f(a(x)) = 3(2x)+1 = 6x+1$, and $eb =
    b \circ e = b(e(x)) = 2(3x)+1 = 6x+1$.
    \item $\boldsymbol{ag \rightarrow fa}$: $ag = g \circ a = g(a(x)) = 3(2x)+2 = 6x+2$, and $fa = a \circ f = a(f(x)) = 2(3x+1) = 6x+2$.
    \item $\boldsymbol{be \rightarrow fb}$: $be = e \circ b = e(b(x)) = 3(2x+1) = 6x+3$, and $fb = b \circ f = b(f(x)) = 2(3x+1)+1 = 6x+3$.
    \item $\boldsymbol{bf \rightarrow ga}$: $bf = f \circ b = f(b(x)) = 3(2x+1)+1 = 6x+4$, and $ga =  a \circ g = a(g(x)) = 2(3x+2) = 6x+4$.
    \item $\boldsymbol{bg \rightarrow gb}$: $bg = g \circ b = g(b(x)) = 3(2x+1)+2 = 6x+5$, and $gb = b \circ g = b(g(x)) = 2(3x+2)+1 = 6x+5$.
\end{itemize}
Hence, these rules are all correct. \par
The $C$ rules take advantage of the fact that the $c$ symbol is a binary 1, and, in a strictly binary string, it is the most significant bit of the corresponding number $x$. When the ternary symbol is adjacent to the $c$ symbol, we apply one of the three $c$ rules to convert the ternary symbol into binary symbol(s). These rules also preserve the number the string represents, and the proofs showing this is the case for each of these rules are shown here:
\begin{itemize}
    \item $\boldsymbol{ce \rightarrow cb}$: $ce = e \circ c = e(c(x)) = 3(1) = 3$, and $cb = b
    \circ c = b(c(x)) = 2(1)+1 = 3$.
    \item $\boldsymbol{cf \rightarrow caa}$: $cf = f \circ c = f(c) = 3(1)+ 1 = 4$, and $caa = a \circ a \circ c = a(a(c(x))) = 2(2(1)) = 4$.
    \item $\boldsymbol{cg \rightarrow cab}$: $cg = g \circ c = g(c) = 3(1)+ 2 = 5$, and $cab = b \circ a \circ c = b(a(c(x))) = 2(2(1))+1 = 5$.
\end{itemize}
Hence, we have shown that the $A$, $B$, and $C$ rules all preserve value, and the $D$ rules correctly apply the $3x+1$ mapping. \par
Here is how one can run the SRS and preserve an ordering we know to be valid:
\begin{enumerate}
    \item Take the initial input number, and convert it to binary. Make the leading 1 a $c$ symbol, and all 0's and other 1's $a$'s and $b$'s, respectively.
    \item Until we have the string $cd$: 
    \begin{itemize}
        \item Apply the appropriate $D$ rule.
        \item If $D_2$ is applied, apply $A$ and $B$ rules until the ternary symbol and the $c$ are adjacent, then apply the appropriate $C$ rule.
    \end{itemize}
\end{enumerate}
This order of applying the SRRs is correct, because one takes a string that is strictly in binary symbols and applies the correct $D$ rules until rule $D_2$ is applied, then it handles the ternary symbol immediately by applying $A$, $B$, and $C$ rules until it is converted into binary symbol(s). The number is not changed during application of these rules, making the ordering correct. This ordering was used in building the system that investigated the number of steps needed in the rewrite system, which will be talked about in section~\ref{sec:hardnessrewriterules}. \par
If we take this SRS and model matrix functions for all symbols that cause all inputs to decrease, then we believe we can prove the Collatz Conjecture.

%We can also attempt to alter the rules if we believe another equivalent set of rules might be easier to prove using the matrix termination methodology. \par
