%THOUGHT: I think it would be far cleaner, instead of the amount of text I threw up into the printed out Thesis, to just go ahead and define both 3x+1 mapping and sequence, but tie them to Algorithm 1, and note the fact that we use 3x+1 mapping and sequence throughout the Thesis (since they are far cleaner than referring to recursive calls).
\chapter{Definitions} \label{sec:defns}
We will use the following terms throughout this thesis:
\begin{itemize}
    \item $\boldsymbol{3N+1}$\textbf{ sequence}: Define this as follows: 
    \begin{align*}
        N_0 = N &= \text{ initial input number} \\
        N_{i+1} &= \begin{cases} 
        N_{i}/2 &\text{ if $N_i$ is even} \\
        3 N_{i} + 1 &\text{ if $N_i$ is odd} \\
        \end{cases}
    \end{align*}
    This sequence can continue for arbitrarily large values of $i$, but we are only interested in following the sequence until $N_i$, for some $i$, is 1, as any value after 1 follows the cycle of $1 \rightarrow 4 \rightarrow 2 \rightarrow 1$ infinitely. \par
    Note that if we run $\Col{N}$, we would, after $i$ recursive calls, end up passing, into the $i+1$\textsuperscript{th} recursive call of Algorithm~\ref{alg:ColR}, $N_{i+1}$. Algorithm~\ref{alg:ColR} can be modified to store the sequence of numbers $N, N_1, \ldots, N_i$, meaning it can effectively compute $3N+1$ sequences as well. However, for simplicity, throughout the remainder of this thesis, we just refer to $3N+1$ sequences, with $N$ as the initial input, and $N_i$ as the $i$ number in the $3N+1$ sequence that started with $N$.
    \item $\boldsymbol{3N+1}$\textbf{ mapping}: One recursive call of $\Col{N}$ for input number $N$, giving us the number $N_1$, as defined in the $3N+1$ sequence.
    \item \textbf{Avoidance set $\boldsymbol A$:} The second parameter of Algorithm~\ref{alg:ColSP}, $\ColMod{N}{A}{b}$, which was defined in the introduction. $A = \{a_1 \ldots a_s\}$ is all of the numbers modulo the positive integer $b$ that cause termination of the algorithm. This termination condition is in addition to the sole termination condition of Algorithm~\ref{alg:ColR}: when the input into $\Col{N}$, $N$, is 1. Note that for all $a \in A$, $0 \le a < b$, and $A = \varnothing$ turns Algorithm~\ref{alg:ColSP} into Algorithm~\ref{alg:ColR}.
    \item \textbf{Collatz Variant:} When we say ``Collatz Variant'' we are referring to a specific instance of Algorithm~\ref{alg:ColSP}.
    \item $\boldsymbol{\ColMod{N}{A}{b}:}$ A specific instance of a Collatz Variant for a positive integer $N$, avoidance set $A$, and positive integer $b$. 
    \item \textbf{Collatz Variant A:} The vast majority of our analysis on Collatz Variants is on instances when $b = 8$, so when we say Collatz Variant $A$, it is shorthand for $\ColMod{N}{A}{8}$. We often list several base 8 instances of Collatz Variants together, so for instance, if we say Collatz Variants 1, 5, 7, and $\{1,5\}$; we mean the instances $\ColMod{N}{\{1\}}{8}$, $\ColMod{N}{\{5\}}{8}$, $\ColMod{N}{\{7\}}{8}$, and $\ColMod{N}{\{1,5\}}{8}$; respectively. Note that when listing variants in this manner, we omit the braces normally around singleton sets.
\item \textbf{Number of bits:} For any number $N$, we say that, written in binary, it has $m$ bits. Note that $m = \log_2{N}$.
\end{itemize}
%%  LocalWords:  th SRS Subproblem
