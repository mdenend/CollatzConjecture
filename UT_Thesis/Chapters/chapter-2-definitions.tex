\chapter{Definitions} \label{sec:defns}
We will use the following terms throughout this Thesis:
\begin{itemize}
    \item $\boldsymbol{3x+1}$\textbf{ mapping}: A one step application of $\Col{N}$ to some input number $N$.
    \item $\boldsymbol{3x+1}$\textbf{ sequence}: Define this as follows: 
    \begin{align*}
        x_0 = x &= \text{ initial input number} \\
        x_{i+1} &= \begin{cases} 
        x_{i}/2 &\text{ if $x_i$ is even} \\
        3 x_{i} + 1 &\text{ if $x_i$ is odd} \\
        \end{cases}
    \end{align*}
    This sequence can continue for arbitrarily large values of $i$, but we are only interested in following the sequence until $x_i$, for some $i$, is 1, as any value after 1 follows the cycle of $1 \rightarrow 4 \rightarrow 2 \rightarrow 1$ infinitely.
    \item \textbf{Avoidance set $\boldsymbol A$:} The second parameter of Algorithm~\ref{alg:ColSP}, which was defined in the introduction. $A = \{a_1 \ldots a_s\}$ is all of the numbers modulo the positive integer $b$ (the third parameter of Algorithm~\ref{alg:ColSP}) that cause termination of the algorithm. This termination condition is in addition to the sole termination condition of Algorithm~\ref{alg:ColR}: when the input number $N$ is 1. Note that for all $a \in A$, $0 \le a < b$, and $A = \varnothing$ turns Algorithm~\ref{alg:ColSP} into Algorithm~\ref{alg:ColR}.
    \item \textbf{Collatz Variant:} When we say ``Collatz Variant'' we are referring to Algorithm~\ref{alg:ColSP}, for some positive integer $N$, avoidance set $A$, and positive integer $b$. There are a couple of shorthands we use throughout this Thesis:
      \begin{itemize}
      \item $\boldsymbol{\ColMod{N}{A}{b}:}$ Parameters are the same as Algorithm~\ref{alg:ColSP}.
      \item \textbf{Collatz Variant A:} The vast majority of our analysis is when $b = 8$, so when we say Collatz Variant $A$, it is shorthand for $\ColMod{N}{A}{8}$. For singleton sets $A$, we omit the braces normally around sets. We often list several Collatz Variants together, so for instance, if we say Collatz Variants 1, 5, 7, and $\{1,5\}$, we mean the variants $\ColMod{N}{\{1\}}{8}$, $\ColMod{N}{\{5\}}{8}$, $\ColMod{N}{\{7\}}{8}$, and $\ColMod{N}{\{1,5\}}{8}$, respectively.
      \end{itemize}
\item \textbf{Collatz String Rewrite System (SRS):} Created by Aaronson~\cite{HeuleAaronson}, and introduced in chapter~\ref{sec:SRSandSAT}, along with background on string rewrite systems, this is used in the latter portion of this paper as an alternative way of expressing the Collatz Conjecture. Also called Aaronson's SRS.
\item \textbf{Number of bits:} For an input number $x$, we say that, written in binary, it has $m$ bits.
\end{itemize}
We also have definitions within Chapter~\ref{sec:subhrdnspred} and~\ref{sec:hardnessrewriterules}. See sections~\ref{subsec:algdefinemeasure} and~\ref{subsec:rewritemeasuredefs}, respectively.
%INSTEAD OF DEFINING SEVERAL THINGS HERE, LET'S DO THE FOLLOWING INSTEAD: Say that these terms will also be used throughout the paper, but will be defined more formally in (give section number)
%Collatz base b Graph
%"Node a:"
%
